\section{Introduction}

Cognitive control is thought to support adaptive behavior by overcoming automatic, impulsive responses. Longstanding theoretical models propose cognitive moderates the relationship between automatic impulses and later behavior \citep{diamond13}. This is also assumed in neural measures, with the P300 event-related potential (ERP) often used to quantify cognitive control. However, no studies have directly tested this moderation. What is also unclear is how this assumed relationship is altered in features associated with deteriorated cognitive control, such as among impulsive individuals and those who engage in heavy or uncontrolled alcohol use. Heavy alcohol use during late adolescence (i.e., college-age) can have deleterious effects on neurological development as critical brain regions are undergoing maturation. This includes the prefrontal cortex, which is implicated in cognitive control function. Rates of alcohol use continue to remain a pervasive issue in the United States. 49.6 \% (16.9 million) of Americans aged reported alcohol use in the last month. This is attributable to many sources, but namely due to the belief that drinking is a core aspect of the college experience \citep{tan12}. Heavy drinking in college is cause for concern due to the neuro-developmental implications. The typical college-age (18-24 years) aligns with the late adolescent period, where essential brain regions are undergoing maturation. The prefrontal cortex (PFC) is just one example of this. Impaired PFC is linked to impulsive behavior such as experimenting with alcohol and other drugs and the PFC is considered is a critical area for cognitive control function. It is also broadly theorized that correlates of cognitive control behave differently among heavy drinkers. However, this model has yet to be tested in this manner. Cognitive control functions as an umbrella of several mechanisms including response inhibition and interference control. Cognitive control is typically measured using tasks that elicit response conflict, including Simon (response inhibition) and Flanker (interference control) tasks. Simon conflict reflects the automatic tendency to respond with the hand ipsilateral to a stimulus regardless of the correct response. Flanker conflict occurs when irrelevant flanking stimuli indicate a response unrelated to the target stimulus. Few studies have examined how different cognitive control mechanisms interact and no studies have explored this interaction in the context of clinical features such as hazardous alcohol use or impulsivity. ERPs reflecting cognitive control and early response preparation will be compared to task performance (reaction time, accuracy) to test the hypotheses that (1) cognitive control (Via P3 ERP amplitude) moderates prepotent response activation and behavior, and (2) because cognitive control is reduced in high binge drinking and impulsivity, the moderation is weakened, leading to a strengthened relationship between impulsive response activation and unsuccessful inhibition (poor task performance). 